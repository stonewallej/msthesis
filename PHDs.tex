\chapter{New contact technologies: The PhDs detector}
\label{ch:New contact technologies: The PhDs detector}
\section{subsection title}
\indent a little bit of an intro  -  mention where its from

The PHD's detector is a cylindrical planar crystal of high purity germanium with a 90mm diameter and 10mm thickness. Like the Ortec detector, it has a total of 32 electrically segmented strips; 2 sets of 16 on either side with one set of contacts running orthogonal to the other such that a 2-dimensional grid of 256 pixels is formed. The contacts have a 5mm pitch and a .25mm spacing. There is a gap of 3.125mm from the edge of the last contact to the edge of the crystal that is not utilized. Both sets of contacts are made with amorphous Ge using a photolithographic process. Thickness ???????. The PHD's detector is mechanically cooled my means of an ion pump to a temperature of 78K.

The "A" side (DC side) of the PHD's detector has contacts running horizontally providing Y axis positioning. These contacts are labeled with numbers 0-15 and are at electrical ground. The A side contacts collect the holes created by an interacting gamma ray. Output appears as a positive pulse. 

The "B" side (AC side) of the detector has vertical contacts providing X axis positioning. B side contacts are labeled 16-31 and are operated at 600V. These contacts collect the electrons created when a gamma interacts with the crystal and output appears as a negative pulse.

The benefits of the new amorphous Ge contacts are twofold. One, as a more stable contact it is believed the amorphous Ge will overcome the problems the lithium contacts have with mobility and thickness. With the thinner amorphous Ge, the spacing needed to electrically isolate the contacts can be much shallower and done with photolithography instead of a physical saw cut. Decreased mobility of the contact material allows for an intra-strip spacing of half the width needed in detectors with lithium contacts.

\section{next subsection}
\indent 




 


 