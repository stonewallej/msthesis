\chapter{Introduction}
\label{ch:Introduction}

\section{Gamma Rays in Nuclear Physics and Applications}
\indent My intro.
Germanium detectors are used in laboratories worldwide for fundamental research in nuclear physics.

\section{Detection Approaches}
\indent There are many methods for collecting time and energy information from nuclear radiation but all detectors use the same basic method.The radiation enters the detector, interacts with the detector material, electrons are removed from their orbits and then collected by electronics \cite{KK88}. When timing information is important, detectors with fast charge collection times are most useful. When accurate energy collection is critical, detectors with a proportional output pulse are needed. Balancing the need for these two detector characteristics is the first step in designing a nuclear physics experiment. Consideration must also be given to the geometry of the detector material and the experimental setup. For different applications, coaxial or planar detectors might be better. If the source needs 4$\pi$ coverage, economical concerns might influence your detector choice. 

say anything about particle energy and stopping power of different detectors?

The simplest type of radiation detector is a gas counter. A gas counter is an ionization chamber filled with a mixture of gas chosen for its low working voltage, high gain, good proportionality, and abliity to handle a high rate \cite{Leo94}. The detector has an electric charge applied to it and essentially behaves as a parallel plate capacitor \cite{KK88}. When the radiation enters the gas counter, it ionizes the gas and the electric field pulls the ions and electrons towards each plate. Radiation intensity is recorded as a current that is proportional to the activity of the source and energy of the radiation \cite{KK88}. To record individual pulses, signal amplification is needed. This is accomplished by increasing the applied voltage, causing the charge carriers to be pulled to the plates more quickly and causing additional collisions with the gas. This creates more ionized atoms, amplifying the signal. Timing from these types of detectors comes from the drift time of the charge carriers; the time it takes for the ion to travel from where it was created to where it gets collected \cite{KK88}.
--tons more in Leo ch 6.

Developed in the 1950s, scintillation detectors are unique in the sense that the electrons formed by ionization are not what forms the detected pulse \cite{KK88}. Instead, radiation entering and interacting with the detector material excites the material's atoms which then emit light \cite{KK88}. This light enters a photomultiplier tube and is collected as a pulse. Scintillation detectors can be made from organic or inorganic materials and can contain activator impurities to increase the probability for photon emission and reduce self-absorption \cite{KK88}. 

Scintillators are some of the most widely used detectors used today because of their sensitivity to energy, fast response, and ability to discriminate pulse shapes. Most scintillators have a linear response to deposited energy making the light output proportional to the energy \cite{Leo94}. While other options exist, scintillation detectors can be used for energy spectrometers. The fast response time allows for obtaining time differences between events and also means less dead time while collecting data \cite{Leo94}. Some scintillators can distinguish between different incident particles by pulse shape discrimination. Scintillators can be organic crystals or liquids, plastic, inorganic crystals, gases and glasses. There are challenges for each type of scintillator material. Crystals can be very brittle and their resolution is dependent on the orientation of the axis while liquids can be extremely sensitive to impurities \cite{Leo94}. Plastics can be damaged by organic solvents and need to be handled with plastic gloves \cite{Leo94}. In general, inorganic crystals have a slower response time and must be used in a protective housing to protect from moisture \cite{Leo94}. Gas scintillators emit a light that is difficult for photomultiplier tubes to detect and glass has low light output.
--Leo ch 7
%move this section to the "Gammas" chapter? %%%%%%%%%%%%%%%%%%%%%%%%%%%
Development of semiconductor detectors really progressed in the 1960's when they became commercailly available. Soon these detectors were being put to use for nuclear physics research. Semiconductor detectors are solid crystals with energy gaps smaller than the gap in insulator materials. The crystal materials, usually germanium or silicon, have higher densities giving a larger interaction probability. Semiconductor detectors can both handle a large electric field and have easily removable electrons that move easily through the material \cite{KK88}. 

Semiconductor detectors create nearly two orders of magnitude more charge carriers than a scintillation detector. This provides the semiconductor detectors greatest advantage, vastly improved energy resolution \cite{Leo94}.

Semiconductor detectors are usually operated at liquid nitrogen temperatures to reduce the noise from the thermal motion of the electrons. Common crystal shapes are coaxial and planar where the charge carriers move radially or laterally to be collected by the contacts. 
%%%%%%%%%%%%%%%%%%%%%%%%%%%%%%%%%%%%%%%%%%%%%%%%%

HPGe
Intrinsic germanium detectors, often called high-purity germanium (HPGe), have impurities of less than 10^10 atoms/cm^3 and only need to be cooled to LN2 temps when a voltage is applied. 

\section{Major Germanium Arrays in Labs Worldwide}
\indent Greta/Gretina, Gammasphere, Euroball...

see proposal

\section{Segmented Detectors}
\indent cost/benifit
electronics to match - risk of cross talk - more money
resolution with large detectors - spatial vs energy resolution

see proposal for this section
 
\section{The Need for New Contact Technologies}
\indent 

see DNP ppt

 