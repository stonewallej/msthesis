\chapter{Introduction}
\label{ch:Introduction}
\section{Inorganic Scintillators}
\indent As the frontiers of nuclear physics are pushed further towards exotic neutron-rich nuclei, and nuclear science becomes an increasing invaluable tool in a diverse range of applications, the need to advance the capabilities of detector technology becomes paramount. The past decade has seen a rapid increase in the development of radiation detectors for nuclear physics, medical physics, nonproliferation, nuclear energy research, and high energy physics. Amongst the various detector types being developed, inorganic scintillators have been ubiquitous throughout nuclear science since Rutherford used ZnS powder to detect $\alpha$-particle backscattering from gold. While scintillators have now been in use for over a century, significant improvement over traditional detectors that have been in use since the 1950's is still being made. 

\indent The most common application of inorganic scintillators is $\gamma$-ray spectroscopy, with a more recent interest in neutron detection. For decades the standard has been thallium activated NaI. However, interest in finding better crystal compounds and activators has lead to the discovery of several scintillators that have been shown to be superior to NaI:Tl in regards to fundamental characteristics such as energy resolution, timing, and efficiency. 

\indent CeBr$_3$ is a recent fast unactivated scintillator which improves and all three aforementioned traits. For the past several decades cerium has become an increasingly popular activator for inorganic scintillators because of its fast decay time and reasonable light yield \cite{GK00}. Because of this, there was interest in scintillators incorporating Ce as one of the primary constituents. CeBr$_3$ may naturally be compared to LaBr$_3$:Ce, another fast high-resolution $\gamma$-ray detector which has been under extensive study. CeBr$_3$ shares many of the same detection characteristics but is slowly attracting more attention for its low intrinsic activity compared to LaBr$_3$:Ce. 

\indent For crystals that incorporate a neutron sensitive isotope, a detector which is capable of measuring both neutrons and $\gamma$-rays can be made. One of the most recently popular and promising of these types is Cs$_2$LiYCl$_6$ (CLYC). CLYC has the unique properties of having good energy and timing resolution, while also being sensitive to both thermal and fast neutrons. Its versatile use as a dual neutron/$\gamma$ detector is facilitated by its ability to clearly discriminate between neutron and $\gamma$-ray events based on pulse shapes. Together, CLYC and CeBr$_3$ represent two of the most interesting new inorganic scintillators whose combined applications span nearly all of nuclear science.

\section{Scintillation Mechanisms}
\indent 




 


 