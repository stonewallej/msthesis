\chapter{Gamma Rays, X-rays and their interaction with matter}
\label{ch:Gamma Rays, X-rays and their interaction with matter}
\section{X-Rays and Gamma Rays}
\indent energy range of gammas and xrays

\section{Conduction band in semiconductors}
\indent Ge crystal, 
insulators, conductors, and semiconductors

Dopants can be added to change the material's conduction by replacing some of the crystal's atoms with dopant atoms that have either one more or one less valence electron. With one more valence electron, the detector is said to be n-type because of the extra negative charge carriers. One less valence electron makes a p-type detector due to the extra positive charge carriers. These positive charge carriers are the absence of electrons and called "holes".

Leo:
energy per electron pair
leakage current


Bringing n- and p-type detectors together the excess electrons and holes combine to create a depletion region of no excess electrons or holes. The electric field created by the diffused charges eventually fixes the size of the depletion region \cite{KK88}. Applying a large reverse-bias (negative bias is applied to the p-type material) forces more excess charge carriers to combine. This increases the size of the depletion region thereby increasing the effective volume of the crystal. 

manufacture methods of HPGe

\section{Three major interactions in matter}
\indent Upon entering the detector material, the gamma ray will most likely undergo one of three processes: photoelectric absorption, Compton scattering, or pair production. Each of these processes dominates in a particular energy range therefore, depending on the type of studies a researcher is engaged in, will have ....... Photoelectric absorption is most likely for a gamma ray up to a few hundred keV and this probability increases with the increasing atomic number of the crystal \cite{GK00}. In photoelectric absorption, a gamma is absorbed by an atom of the detector's crystal and an atomic electron is released \cite{KK88}. The energy of the electron is the incident gamma's energy less the binding energy of the electron. This process is ideal for gamma spectroscopy because the total energy of the gamma ray is observed interacting with the crystal thereby giving a true account of the interaction. 

\indent The second interaction process is Compton scatering where the gamma ray only deposits a fraction of its energy in the detector crystal. The gamma ray then continues on to either interact with the crystal again or escape the active region entierly. Compton scattering is the dominant process for gammas with energies from 150 keV to 4 MeV \cite{KK88}. Because of its fractional energy deposition, identification of these events is very important. The Compton gammas are scattered through an angle according to the Compton scattering formula. Where $E'_\gamma$ is the energy of the scatered gamma, $E_\gamma$ is the energy of the incident gamma, and $\theta$ is the scattering angle. 
%need a label or citation KK88?, add space after the eqn, figure 7.6 from Krane or the one I used in the proposal
\begin{equation}
E^\prime_\gamma = \frac{E_\gamma}{1 + \big(E_\gamma/mc^2\big)\big(1-cos\theta\big)}
\end{equation}
Compton scattered events that exit the detector crystal without distributing their total energy contribute to the background continuum seen in spectra. In a single gamma spectra, the Compton continuum ranges from zero to the Compton edge. This corresponds to scattering angles of 0$^\circ$ and 180$^\circ$ respectively. The angular distribution of scattered gamma rays in a Compton scattering process is given by the Klein-Nishina formula for the differential scattering cross section \cite{KK88}.
\begin{equation}
\frac{d\sigma_c}{d\Omega}=r_o^2\bigg[\frac{1}{1+\alpha(1-cos\theta)}\bigg]^3\bigg[\frac{1+cos\theta}{2}\bigg]\bigg[1+\frac{\alpha^2(1-cos\theta)^2}{(1+cos^2\theta)(1+\alpha(1-cos\theta))}\bigg]
\end{equation}

The energies of multiple events identified as Compton events can be summed in software to give a more complete picture of the actual interaction.  The detection and tracking of these Compton scattering events can be used to either reconstruct a true photopeak from more than one interaction of a single photon in the detector, or use the specific characteristics of Compton-scattered events to measure quantities such as the linear polarization of the incident photon. 

In the third process, pair production, the gamma ray is absorbed and an electron-positron pair is produced. There is a threshold of $2mc^2$, or 1.022MeV, for this process and therefore only likely for gammas of high energy \cite{KK88}.

%show figure 7.8 in krane?

\section{Applying a reverse bias voltage}
\indent depletion region, movement of electrons and holes

All crystalline materials have a structure with th...



Semiconductor detectors have a crystalline structure with their atoms arranged in energy bands. The valence band has tightly bound electrons that stay in place in the lattice while the conduction band has electrons that are deteched from the atoms and free to move \cite{Leo}. Between the valence band and the conduction band is the energy gap. The energy gap has no available energy levels for electrons to move to and its width depends on temperature and pressure. In a 

In an insulator, the energy gap is very wide, prohibiting movement of electrons through the crystal. In conductors, the valence and conduction bands overlap making electron movement, and therefore current flow, easy. Semiconductors are in the middle with a gap of medium size. Thermal excitation of electrons might still promote the electrons into the conduction band when a current is applied. This is the reason a semiconductor detector is operated at liquid nitrogen temperatures. Cooling makes the electrons stay in the valence band and the conductivity of the material decreases. 

electrons and holes

At a particular temperature and applied voltage, the electrons and holes will have a specific drift velocity. As the electrons are excited out of their positions in the lattice, they are promoted into the conduction band where the applied voltage sweeps them towards one of the contacts. The spaces left behind, the holes, behave as positive charges and move the opposite direction. 

Problems with the charge collection may occur when the separated electons and holes recombine due to impurities in the crystal. The charge carriers become trapped in intermediate energy levels in the energy gap and may either be released after some time-causing a prolonged collection time-or recombine and not be collected at all. 
 


 